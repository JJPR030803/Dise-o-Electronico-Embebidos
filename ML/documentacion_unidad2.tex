\documentclass[12pt,letterpaper]{report}
\usepackage[spanish]{babel}
\usepackage[utf8]{inputenc}
\usepackage{graphicx}
\usepackage{hyperref}
\usepackage{listings}
\usepackage{xcolor}
\usepackage{geometry}
\usepackage{titlesec}
\usepackage{fancyhdr}
\usepackage{tocloft}
\usepackage{bookmark}

% Configuración de geometría de página
\geometry{margin=2.5cm}

% Configuración de colores
\definecolor{codegreen}{rgb}{0,0.6,0}
\definecolor{codegray}{rgb}{0.5,0.5,0.5}
\definecolor{codepurple}{rgb}{0.58,0,0.82}
\definecolor{backcolour}{rgb}{0.95,0.95,0.92}

% Configuración de listings para código
\lstdefinestyle{mystyle}{
    backgroundcolor=\color{backcolour},   
    commentstyle=\color{codegreen},
    keywordstyle=\color{magenta},
    numberstyle=\tiny\color{codegray},
    stringstyle=\color{codepurple},
    basicstyle=\ttfamily\footnotesize,
    breakatwhitespace=false,         
    breaklines=true,                 
    captionpos=b,                    
    keepspaces=true,                 
    numbers=left,                    
    numbersep=5pt,                  
    showspaces=false,                
    showstringspaces=false,
    showtabs=false,                  
    tabsize=2
}

\lstset{style=mystyle}

% Configuración de encabezados y pies de página
\pagestyle{fancy}
\fancyhf{}
\fancyhead[L]{Modelos ARIMA y SARIMA}
\fancyhead[R]{\thepage}
\fancyfoot[C]{Universidad Autónoma de Tamaulipas}

% Configuración de la tabla de contenidos
\renewcommand{\cftchapfont}{\bfseries}
\renewcommand{\cftchappagefont}{\bfseries}
\renewcommand{\cftchapdotsep}{\cftdotsep}
\setlength{\cftbeforechapskip}{1em}
\setlength{\cftbeforesecskip}{0.5em}
\setlength{\cftbeforesubsecskip}{0.25em}
\renewcommand{\cftchapafterpnum}{\vskip5pt}
\renewcommand{\cftsecafterpnum}{\vskip3pt}

% Configuración de bookmarks para PDF
\bookmarksetup{
  numbered,
  open,
  depth=3
}

\begin{document}

% Portada
\begin{titlepage}
    \centering
    \vspace*{1cm}
    {\Huge\bfseries Documentación de Modelos ARIMA y SARIMA\\para Análisis de Series Temporales\par}
    \vspace{2cm}
    {\Large\itshape Diseño de Sistemas Embebidos\par}
    \vspace{3cm}

    % Aquí debes insertar el logo de tu escuela
    % Reemplaza "logo_escuela.png" con la ruta a tu logo
    % Comentado para permitir la compilación sin el archivo de logo
    % \includegraphics[width=0.5\textwidth]{logo_uat}\\

    \vspace{3cm}

    {\Large\bfseries Universidad Autónoma de Tamaulipas\par}
    \vspace{0.5cm}
    {\large Facultad de Ingeniería\par}
    \vspace{1.5cm}

    % Aquí debes incluir los nombres de los miembros del equipo
    {\large\bfseries Equipo:\par}
    \vspace{0.5cm}
    {\large Juan Julián Paniagua Rico - a2213332303\par}
    {\large Isaac Sayeg Posadas Perez - a2213332197\par}
    {\large Jorge Roberto García Azzua - a2221335006\par}
    \vspace{1.5cm}

    {\large \today\par}
\end{titlepage}

% Tabla de contenidos

\thispagestyle{fancy}

\newpage

\chapter{Introducción}

\section{Descripción General}
Este documento proporciona una documentación detallada de los modelos ARIMA (AutoRegressive Integrated Moving Average) y SARIMA (Seasonal AutoRegressive Integrated Moving Average) implementados para el análisis y predicción de series temporales. Estos modelos son ampliamente utilizados en estadística y aprendizaje automático para modelar y predecir datos de series temporales.

Los modelos ARIMA y SARIMA son técnicas estadísticas que utilizan observaciones pasadas para predecir valores futuros en una serie temporal. ARIMA se utiliza para series temporales no estacionales, mientras que SARIMA extiende ARIMA para manejar patrones estacionales en los datos.

\section{Características Principales}
Las implementaciones de ARIMA y SARIMA ofrecen las siguientes características principales:

\begin{itemize}
    \item Ajuste (fitting) de modelos a datos de series temporales
    \item Predicción de valores futuros
    \item Evaluación del rendimiento del modelo
    \item Diagnóstico visual mediante gráficos
    \item Búsqueda en malla para optimización de hiperparámetros
    \item Exportación de resultados a archivos CSV
    \item Descomposición de series temporales (solo en SARIMA)
\end{itemize}

Esta documentación incluye explicaciones detalladas de las clases, métodos y funcionalidades implementadas, así como ejemplos de uso y recomendaciones para la aplicación efectiva de estos modelos.

\chapter{Modelo ARIMA}

\section{Descripción General}
ARIMA (AutoRegressive Integrated Moving Average) es un modelo estadístico utilizado para analizar y predecir datos de series temporales. El modelo ARIMA combina tres componentes:

\begin{itemize}
    \item \textbf{AR (AutoRegressive):} Utiliza la relación entre una observación y un número específico de observaciones retrasadas.
    \item \textbf{I (Integrated):} Aplica diferenciación para hacer que la serie temporal sea estacionaria.
    \item \textbf{MA (Moving Average):} Utiliza la dependencia entre una observación y un error residual de un modelo de media móvil aplicado a observaciones anteriores.
\end{itemize}

La implementación de ARIMA en este proyecto se realiza a través de la clase \texttt{ArimaModel}, que proporciona una interfaz completa para trabajar con modelos ARIMA.

\section{Clase ArimaModel}
La clase \texttt{ArimaModel} encapsula toda la funcionalidad necesaria para implementar y gestionar modelos ARIMA para el análisis y predicción de series temporales.

\subsection{Constructor}
\begin{lstlisting}[language=python]
def __init__(self, data=None, order=(1, 0, 0)):
    """
    Inicializa un modelo ARIMA con los parámetros especificados.

    Args:
        data (array-like, optional): Serie temporal para entrenar el modelo.
        order (tuple, optional): Orden del modelo ARIMA (p,d,q).
            p: Términos autorregresivos
            d: Diferenciación necesaria para hacer la serie estacionaria
            q: Términos de media móvil
    """
    self.data = data
    self.order = order
    self.model = None
    self.fitted_model = None
    self.predictions = None
    self.residuals = None
    self.results_manager = None
\end{lstlisting}

El constructor inicializa un modelo ARIMA con los datos y el orden especificados. El orden es una tupla (p, d, q) donde:
\begin{itemize}
    \item p: Número de términos autorregresivos
    \item d: Grado de diferenciación
    \item q: Número de términos de media móvil
\end{itemize}

\subsection{Método fit}
\begin{lstlisting}[language=python]
def fit(self, data=None):
    """
    Entrena el modelo ARIMA con los datos proporcionados.

    Args:
        data (array-like, optional): Serie temporal para entrenar el modelo.
            Si no se proporciona, se utilizan los datos del constructor.

    Returns:
        self: La instancia del modelo entrenado.
    """
    if data is not None:
        self.data = data

    if self.data is None:
        raise ValueError("No se han proporcionado datos para entrenar el modelo")

    self.model = ARIMA(self.data, order=self.order)
    self.fitted_model = self.model.fit()
    self.residuals = self.fitted_model.resid

    return self
\end{lstlisting}

Este método entrena el modelo ARIMA con los datos proporcionados. Si no se proporcionan datos, utiliza los datos pasados al constructor. Devuelve la instancia del modelo para permitir el encadenamiento de métodos.

\subsection{Método predict}
\begin{lstlisting}[language=python]
def predict(self, steps=1):
    """
    Realiza predicciones con el modelo ARIMA entrenado.

    Args:
        steps (int, optional): Número de pasos a predecir. Por defecto es 1.

    Returns:
        array: Predicciones del modelo.
    """
    if self.fitted_model is None:
        raise ValueError("El modelo debe ser entrenado antes de realizar predicciones")

    # Realizar predicción
    self.predictions = self.fitted_model.forecast(steps=steps)
    return self.predictions
\end{lstlisting}

Este método realiza predicciones con el modelo ARIMA entrenado. El parámetro \texttt{steps} especifica el número de pasos futuros a predecir.

\subsection{Método evaluate}
\begin{lstlisting}[language=python]
def evaluate(self):
    """
    Evalúa el rendimiento del modelo calculando métricas comunes.

    Returns:
        dict: Diccionario con las métricas de evaluación.
    """
    if self.fitted_model is None:
        raise ValueError("El modelo debe ser entrenado antes de evaluarlo")

    # Calcular predicciones en muestra
    in_sample_predictions = self.fitted_model.fittedvalues

    # Calcular errores
    residuals = self.data - in_sample_predictions
    mse = np.mean(residuals ** 2)
    rmse = np.sqrt(mse)
    mae = np.mean(np.abs(residuals))

    # Crear diccionario de resultados
    metrics = {
        'AIC': self.fitted_model.aic,
        'BIC': self.fitted_model.bic,
        'MSE': mse,
        'RMSE': rmse,
        'MAE': mae
    }

    return metrics
\end{lstlisting}

Este método evalúa el rendimiento del modelo calculando métricas comunes como AIC (Criterio de Información de Akaike), BIC (Criterio de Información Bayesiano), MSE (Error Cuadrático Medio), RMSE (Raíz del Error Cuadrático Medio) y MAE (Error Absoluto Medio).

\subsection{Método plot\_diagnostics}
\begin{lstlisting}[language=python]
def plot_diagnostics(self, figsize=(12, 8)):
    """
    Genera gráficos de diagnóstico para el modelo ARIMA.

    Args:
        figsize (tuple, optional): Tamaño de la figura.
    """
    if self.fitted_model is None:
        raise ValueError("El modelo debe ser entrenado antes de generar diagnósticos")

    fig, axes = plt.subplots(2, 2, figsize=figsize)

    # Gráfico de la serie original y las predicciones
    axes[0, 0].plot(self.data, label='Observado')
    axes[0, 0].plot(self.fitted_model.fittedvalues, color='red', label='Predicciones')
    axes[0, 0].set_title('Valores observados vs predicciones')
    axes[0, 0].legend()

    # Gráfico de residuos
    axes[0, 1].plot(self.residuals)
    axes[0, 1].set_title('Residuos')
    axes[0, 1].axhline(y=0, color='red', linestyle='--')

    # ACF de residuos
    plot_acf(self.residuals, ax=axes[1, 0], lags=20)
    axes[1, 0].set_title('ACF de residuos')

    # PACF de residuos
    plot_pacf(self.residuals, ax=axes[1, 1], lags=20)
    axes[1, 1].set_title('PACF de residuos')

    plt.tight_layout()
    plt.show()
\end{lstlisting}

Este método genera gráficos de diagnóstico para el modelo ARIMA, incluyendo:
\begin{itemize}
    \item Valores observados vs predicciones
    \item Residuos
    \item Función de Autocorrelación (ACF) de residuos
    \item Función de Autocorrelación Parcial (PACF) de residuos
\end{itemize}

\subsection{Método plot\_forecast}
\begin{lstlisting}[language=python]
def plot_forecast(self, steps=10, alpha=0.05, figsize=(10, 6)):
    """
    Visualiza las predicciones futuras con intervalos de confianza.

    Args:
        steps (int, optional): Número de pasos a predecir. Por defecto es 10.
        alpha (float, optional): Nivel de significancia para los intervalos de confianza.
        figsize (tuple, optional): Tamaño de la figura.
    """
    if self.fitted_model is None:
        raise ValueError("El modelo debe ser entrenado antes de visualizar predicciones")

    # Realizar predicción
    forecast_result = self.fitted_model.get_forecast(steps=steps)
    forecast_index = np.arange(len(self.data), len(self.data) + steps)

    # Obtener predicciones e intervalos de confianza
    forecast_mean = forecast_result.predicted_mean
    conf_int = forecast_result.conf_int(alpha=alpha)

    # Crear figura
    plt.figure(figsize=figsize)

    # Graficar datos históricos
    plt.plot(np.arange(len(self.data)), self.data, label='Observado')

    # Graficar predicciones
    plt.plot(forecast_index, forecast_mean, color='red', label='Predicción')

    # Graficar intervalos de confianza
    plt.fill_between(forecast_index,
                     conf_int.iloc[:, 0],
                     conf_int.iloc[:, 1],
                     color='pink', alpha=0.3, label=f'Intervalo de confianza {(1 - alpha) * 100}%')

    plt.title('Pronóstico ARIMA')
    plt.legend()
    plt.grid(True)
    plt.show()
\end{lstlisting}

Este método visualiza las predicciones futuras con intervalos de confianza. Muestra los datos históricos, las predicciones y los intervalos de confianza en un gráfico.

\subsection{Método grid\_search}
\begin{lstlisting}[language=python]
def grid_search(self, data=None, p_range=(0, 2), d_range=(0, 2), q_range=(0, 2)):
    """
    Realiza una búsqueda en malla para encontrar los mejores hiperparámetros (p,d,q).

    Args:
        data (array-like, optional): Serie temporal. Si no se proporciona, se usan los datos del constructor.
        p_range (tuple, optional): Rango de valores para p.
        d_range (tuple, optional): Rango de valores para d.
        q_range (tuple, optional): Rango de valores para q.

    Returns:
        tuple: La mejor configuración (p,d,q) encontrada.
    """
    if data is not None:
        self.data = data

    if self.data is None:
        raise ValueError("No se han proporcionado datos para la búsqueda en malla")

    # Inicializar ResultsManager para guardar resultados
    results = []
    self.results_manager = ResultsManager(results)

    best_aic = float('inf')
    best_order = None

    # Iterar sobre todas las combinaciones
    for p in range(p_range[0], p_range[1] + 1):
        for d in range(d_range[0], d_range[1] + 1):
            for q in range(q_range[0], q_range[1] + 1):
                try:
                    # Crear y ajustar modelo
                    model = ARIMA(self.data, order=(p, d, q))
                    model_fit = model.fit()

                    # Guardar resultados
                    result = ResultManager(
                        va=model_fit.aic,
                        vo=best_aic,
                        iteracion=p * 100 + d * 10 + q,
                        modelo=f"ARIMA({p},{d},{q})"
                    )
                    self.results_manager.guardar_dato(result)

                    # Actualizar el mejor modelo encontrado
                    if model_fit.aic < best_aic:
                        best_aic = model_fit.aic
                        best_order = (p, d, q)

                    print(f"ARIMA({p},{d},{q}) - AIC: {model_fit.aic}")

                except Exception as e:
                    print(f"Error en ARIMA({p},{d},{q}): {str(e)}")

    # Actualizar los parámetros del modelo con la mejor configuración
    self.order = best_order

    # Entrenar el modelo con los mejores parámetros
    self.fit()

    return best_order
\end{lstlisting}

Este método realiza una búsqueda en malla para encontrar los mejores hiperparámetros (p, d, q) para el modelo ARIMA. Prueba diferentes combinaciones de parámetros y selecciona la que produce el menor valor de AIC.

\subsection{Método to\_csv}
\begin{lstlisting}[language=python]
def to_csv(self, filepath):
    """
    Guarda los resultados de la búsqueda en malla en un archivo CSV.

    Args:
        filepath (str): Ruta donde guardar el archivo CSV.
    """
    if self.results_manager:
        self.results_manager.to_csv(filepath)
    else:
        print("No hay resultados para guardar")
\end{lstlisting}

Este método guarda los resultados de la búsqueda en malla en un archivo CSV.

\subsection{Método summary}
\begin{lstlisting}[language=python]
def summary(self):
    """
    Devuelve un resumen del modelo ARIMA.

    Returns:
        str: Resumen del modelo.
    """
    if self.fitted_model is None:
        raise ValueError("El modelo debe ser entrenado antes de generar el resumen")

    return self.fitted_model.summary()
\end{lstlisting}

Este método devuelve un resumen del modelo ARIMA, incluyendo estadísticas y parámetros estimados.

\chapter{Modelo SARIMA}

\section{Descripción General}
SARIMA (Seasonal AutoRegressive Integrated Moving Average) es una extensión del modelo ARIMA que incorpora componentes estacionales. Además de los componentes AR, I y MA, SARIMA incluye sus equivalentes estacionales:

\begin{itemize}
    \item \textbf{SAR (Seasonal AutoRegressive):} Componente autorregresivo estacional.
    \item \textbf{SI (Seasonal Integrated):} Diferenciación estacional.
    \item \textbf{SMA (Seasonal Moving Average):} Componente de media móvil estacional.
\end{itemize}

La implementación de SARIMA en este proyecto se realiza a través de la clase \texttt{SarimaModel}, que proporciona una interfaz completa para trabajar con modelos SARIMA.

\section{Clase SarimaModel}
La clase \texttt{SarimaModel} encapsula toda la funcionalidad necesaria para implementar y gestionar modelos SARIMA para el análisis y predicción de series temporales con patrones estacionales.

\subsection{Constructor}
\begin{lstlisting}[language=python]
def __init__(self, data=None, order=(1, 0, 0), seasonal_order=(0, 0, 0, 0)):
    """
    Inicializa un modelo SARIMA con los parámetros especificados.

    Args:
        data (array-like, optional): Serie temporal para entrenar el modelo.
        order (tuple, optional): Orden del modelo ARIMA (p,d,q).
            p: Términos autorregresivos
            d: Diferenciación necesaria para hacer la serie estacionaria
            q: Términos de media móvil
        seasonal_order (tuple, optional): Orden estacional (P,D,Q,s).
            P: Términos autorregresivos estacionales
            D: Diferenciación estacional
            Q: Términos de media móvil estacionales
            s: Período estacional
    """
    self.data = data
    self.order = order
    self.seasonal_order = seasonal_order
    self.model = None
    self.fitted_model = None
    self.predictions = None
    self.residuals = None
    self.results_manager = None
\end{lstlisting}

El constructor inicializa un modelo SARIMA con los datos, el orden y el orden estacional especificados. El orden es una tupla (p, d, q) y el orden estacional es una tupla (P, D, Q, s) donde:
\begin{itemize}
    \item P: Número de términos autorregresivos estacionales
    \item D: Grado de diferenciación estacional
    \item Q: Número de términos de media móvil estacional
    \item s: Período estacional
\end{itemize}

\subsection{Método fit}
\begin{lstlisting}[language=python]
def fit(self, data=None, exog=None):
    """
    Entrena el modelo SARIMA con los datos proporcionados.

    Args:
        data (array-like, optional): Serie temporal para entrenar el modelo.
            Si no se proporciona, se utilizan los datos del constructor.
        exog (array-like, optional): Variables exógenas para la regresión.

    Returns:
        self: La instancia del modelo entrenado.
    """
    if data is not None:
        self.data = data

    if self.data is None:
        raise ValueError("No se han proporcionado datos para entrenar el modelo")

    self.model = SARIMAX(
        self.data,
        exog=exog,
        order=self.order,
        seasonal_order=self.seasonal_order,
        enforce_stationarity=False,
        enforce_invertibility=False
    )

    self.fitted_model = self.model.fit(disp=False)
    self.residuals = self.fitted_model.resid

    return self
\end{lstlisting}

Este método entrena el modelo SARIMA con los datos proporcionados. Si no se proporcionan datos, utiliza los datos pasados al constructor. También permite incluir variables exógenas para la regresión. Devuelve la instancia del modelo para permitir el encadenamiento de métodos.

\subsection{Método predict}
\begin{lstlisting}[language=python]
def predict(self, steps=1, exog=None):
    """
    Realiza predicciones con el modelo SARIMA entrenado.

    Args:
        steps (int, optional): Número de pasos a predecir. Por defecto es 1.
        exog (array-like, optional): Variables exógenas para la predicción.

    Returns:
        array: Predicciones del modelo.
    """
    if self.fitted_model is None:
        raise ValueError("El modelo debe ser entrenado antes de realizar predicciones")

    # Realizar predicción
    self.predictions = self.fitted_model.forecast(steps=steps, exog=exog)
    return self.predictions
\end{lstlisting}

Este método realiza predicciones con el modelo SARIMA entrenado. El parámetro \texttt{steps} especifica el número de pasos futuros a predecir. También permite incluir variables exógenas para la predicción.

\subsection{Método evaluate}
\begin{lstlisting}[language=python]
def evaluate(self):
    """
    Evalúa el rendimiento del modelo calculando métricas comunes.

    Returns:
        dict: Diccionario con las métricas de evaluación.
    """
    if self.fitted_model is None:
        raise ValueError("El modelo debe ser entrenado antes de evaluarlo")

    # Calcular predicciones en muestra
    in_sample_predictions = self.fitted_model.fittedvalues

    # Calcular errores
    residuals = self.data - in_sample_predictions
    mse = np.mean(residuals ** 2)
    rmse = np.sqrt(mse)
    mae = np.mean(np.abs(residuals))

    # Crear diccionario de resultados
    metrics = {
        'AIC': self.fitted_model.aic,
        'BIC': self.fitted_model.bic,
        'MSE': mse,
        'RMSE': rmse,
        'MAE': mae
    }

    return metrics
\end{lstlisting}

Este método evalúa el rendimiento del modelo calculando métricas comunes como AIC, BIC, MSE, RMSE y MAE.

\subsection{Método plot\_diagnostics}
\begin{lstlisting}[language=python]
def plot_diagnostics(self, figsize=(16, 12)):
    """
    Genera gráficos de diagnóstico para el modelo SARIMA.

    Args:
        figsize (tuple, optional): Tamaño de la figura.
    """
    if self.fitted_model is None:
        raise ValueError("El modelo debe ser entrenado antes de generar diagnósticos")

    fig, axes = plt.subplots(3, 2, figsize=figsize)

    # Gráfico de la serie original y las predicciones
    axes[0, 0].plot(self.data, label='Observado')
    axes[0, 0].plot(self.fitted_model.fittedvalues, color='red', label='Predicciones')
    axes[0, 0].set_title('Valores observados vs predicciones')
    axes[0, 0].legend()

    # Gráfico de residuos
    axes[0, 1].plot(self.residuals)
    axes[0, 1].set_title('Residuos')
    axes[0, 1].axhline(y=0, color='red', linestyle='--')

    # ACF de residuos
    plot_acf(self.residuals, ax=axes[1, 0], lags=40)
    axes[1, 0].set_title('ACF de residuos')

    # PACF de residuos
    plot_pacf(self.residuals, ax=axes[1, 1], lags=40)
    axes[1, 1].set_title('PACF de residuos')

    # Histograma de residuos
    axes[2, 0].hist(self.residuals, bins=25)
    axes[2, 0].set_title('Histograma de residuos')

    # QQ plot de residuos
    from scipy import stats
    stats.probplot(self.residuals, dist="norm", plot=axes[2, 1])
    axes[2, 1].set_title('QQ plot de residuos')

    plt.tight_layout()
    plt.show()
\end{lstlisting}

Este método genera gráficos de diagnóstico para el modelo SARIMA, incluyendo:
\begin{itemize}
    \item Valores observados vs predicciones
    \item Residuos
    \item Función de Autocorrelación (ACF) de residuos
    \item Función de Autocorrelación Parcial (PACF) de residuos
    \item Histograma de residuos
    \item QQ plot de residuos
\end{itemize}

\subsection{Método plot\_forecast}
\begin{lstlisting}[language=python]
def plot_forecast(self, steps=10, alpha=0.05, figsize=(12, 6), exog=None):
    """
    Visualiza las predicciones futuras con intervalos de confianza.

    Args:
        steps (int, optional): Número de pasos a predecir. Por defecto es 10.
        alpha (float, optional): Nivel de significancia para los intervalos de confianza.
        figsize (tuple, optional): Tamaño de la figura.
        exog (array-like, optional): Variables exógenas para la predicción.
    """
    if self.fitted_model is None:
        raise ValueError("El modelo debe ser entrenado antes de visualizar predicciones")

    # Realizar predicción
    forecast_result = self.fitted_model.get_forecast(steps=steps, exog=exog)
    forecast_index = np.arange(len(self.data), len(self.data) + steps)

    # Obtener predicciones e intervalos de confianza
    forecast_mean = forecast_result.predicted_mean
    conf_int = forecast_result.conf_int(alpha=alpha)

    # Crear figura
    plt.figure(figsize=figsize)

    # Graficar datos históricos
    plt.plot(np.arange(len(self.data)), self.data, label='Observado')

    # Graficar predicciones
    plt.plot(forecast_index, forecast_mean, color='red', label='Predicción')

    # Graficar intervalos de confianza
    plt.fill_between(forecast_index,
                     conf_int.iloc[:, 0],
                     conf_int.iloc[:, 1],
                     color='pink', alpha=0.3, label=f'Intervalo de confianza {(1 - alpha) * 100}%')

    plt.title('Pronóstico SARIMA')
    plt.legend()
    plt.grid(True)
    plt.show()
\end{lstlisting}

Este método visualiza las predicciones futuras con intervalos de confianza. Muestra los datos históricos, las predicciones y los intervalos de confianza en un gráfico. A diferencia del método equivalente en ARIMA, este método también permite incluir variables exógenas para la predicción.

\subsection{Método grid\_search}
\begin{lstlisting}[language=python]
def grid_search(self, data=None, p_range=(0, 2), d_range=(0, 1), q_range=(0, 2),
                P_range=(0, 1), D_range=(0, 1), Q_range=(0, 1), s_values=[12]):
    """
    Realiza una búsqueda en malla para encontrar los mejores hiperparámetros.

    Args:
        data (array-like, optional): Serie temporal. Si no se proporciona, se usan los datos del constructor.
        p_range (tuple): Rango de valores para p (componente AR).
        d_range (tuple): Rango de valores para d (componente I).
        q_range (tuple): Rango de valores para q (componente MA).
        P_range (tuple): Rango de valores para P (componente AR estacional).
        D_range (tuple): Rango de valores para D (componente I estacional).
        Q_range (tuple): Rango de valores para Q (componente MA estacional).
        s_values (list): Lista de valores estacionales a probar.

    Returns:
        tuple: La mejor configuración (p,d,q)(P,D,Q,s) encontrada.
    """
    if data is not None:
        self.data = data

    if self.data is None:
        raise ValueError("No se han proporcionado datos para la búsqueda en malla")

    # Inicializar ResultsManager para guardar resultados
    results = []
    self.results_manager = ResultsManager(results)

    best_aic = float('inf')
    best_params = None

    # Contador para iteración
    iteration_counter = 0

    # Iterar sobre todas las combinaciones
    for p in range(p_range[0], p_range[1] + 1):
        for d in range(d_range[0], d_range[1] + 1):
            for q in range(q_range[0], q_range[1] + 1):
                for P in range(P_range[0], P_range[1] + 1):
                    for D in range(D_range[0], D_range[1] + 1):
                        for Q in range(Q_range[0], Q_range[1] + 1):
                            for s in s_values:
                                iteration_counter += 1

                                # Validar combinación
                                if p == 0 and q == 0 and P == 0 and Q == 0:
                                    continue

                                # Mostrar progreso
                                print(f"Evaluando SARIMA({p},{d},{q})({P},{D},{Q},{s})")

                                try:
                                    # Crear y ajustar modelo
                                    model = SARIMAX(
                                        self.data,
                                        order=(p, d, q),
                                        seasonal_order=(P, D, Q, s),
                                        enforce_stationarity=False,
                                        enforce_invertibility=False
                                    )
                                    model_fit = model.fit(disp=False)

                                    # Guardar resultados
                                    result = ResultManager(
                                        va=model_fit.aic,
                                        vo=best_aic,
                                        iteracion=iteration_counter,
                                        modelo=f"SARIMA({p},{d},{q})({P},{D},{Q},{s})"
                                    )
                                    self.results_manager.guardar_dato(result)

                                    # Actualizar el mejor modelo encontrado
                                    if model_fit.aic < best_aic:
                                        best_aic = model_fit.aic
                                        best_params = (p, d, q, P, D, Q, s)

                                    print(f"SARIMA({p},{d},{q})({P},{D},{Q},{s}) - AIC: {model_fit.aic}")

                                except Exception as e:
                                    print(f"Error en SARIMA({p},{d},{q})({P},{D},{Q},{s}): {str(e)}")

    # Actualizar los parámetros del modelo con la mejor configuración
    if best_params:
        p, d, q, P, D, Q, s = best_params
        self.order = (p, d, q)
        self.seasonal_order = (P, D, Q, s)

        # Entrenar el modelo con los mejores parámetros
        self.fit()

    return best_params
\end{lstlisting}

Este método realiza una búsqueda en malla para encontrar los mejores hiperparámetros para el modelo SARIMA. A diferencia del método equivalente en ARIMA, este método también busca los mejores valores para los componentes estacionales (P, D, Q, s). Prueba diferentes combinaciones de parámetros y selecciona la que produce el menor valor de AIC.

\subsection{Método to\_csv}
\begin{lstlisting}[language=python]
def to_csv(self, filepath):
    """
    Guarda los resultados de la búsqueda en malla en un archivo CSV.

    Args:
        filepath (str): Ruta donde guardar el archivo CSV.
    """
    if self.results_manager:
        self.results_manager.to_csv(filepath)
    else:
        print("No hay resultados para guardar")
\end{lstlisting}

Este método guarda los resultados de la búsqueda en malla en un archivo CSV.

\subsection{Método summary}
\begin{lstlisting}[language=python]
def summary(self):
    """
    Devuelve un resumen del modelo SARIMA.

    Returns:
        str: Resumen del modelo.
    """
    if self.fitted_model is None:
        raise ValueError("El modelo debe ser entrenado antes de generar el resumen")

    return self.fitted_model.summary()
\end{lstlisting}

Este método devuelve un resumen del modelo SARIMA, incluyendo estadísticas y parámetros estimados.

\subsection{Método decompose}
\begin{lstlisting}[language=python]
def decompose(self, type='additive'):
    """
    Descompone la serie temporal en sus componentes de tendencia,
    estacionalidad y residuos.

    Args:
        type (str): Tipo de descomposición ('additive' o 'multiplicative').

    Returns:
        object: Resultado de la descomposición.
    """
    from statsmodels.tsa.seasonal import seasonal_decompose

    if self.data is None:
        raise ValueError("No hay datos para descomponer")

    # Crear un índice para la descomposición
    index = pd.date_range(start='2000-01-01', periods=len(self.data), freq='D')
    series = pd.Series(self.data, index=index)

    # Descomponer la serie
    decomposition = seasonal_decompose(series, model=type, period=self.seasonal_order[3])

    # Graficar los resultados
    fig, axes = plt.subplots(4, 1, figsize=(12, 10), sharex=True)

    axes[0].plot(decomposition.observed)
    axes[0].set_title('Serie Original')

    axes[1].plot(decomposition.trend)
    axes[1].set_title('Tendencia')

    axes[2].plot(decomposition.seasonal)
    axes[2].set_title('Estacionalidad')

    axes[3].plot(decomposition.resid)
    axes[3].set_title('Residuos')

    plt.tight_layout()
    plt.show()

    return decomposition
\end{lstlisting}

Este método descompone la serie temporal en sus componentes de tendencia, estacionalidad y residuos. Utiliza la función \texttt{seasonal\_decompose} de statsmodels para realizar la descomposición y visualiza los resultados en un gráfico. Este método es exclusivo de la clase \texttt{SarimaModel} y no está presente en la clase \texttt{ArimaModel}.

\chapter{Ejemplos de Uso}

\section{Ejemplo de Uso de ARIMA}
A continuación se muestra un ejemplo básico de cómo utilizar la clase \texttt{ArimaModel} para analizar y predecir una serie temporal:

\begin{lstlisting}[language=python]
import numpy as np
import pandas as pd
import matplotlib.pyplot as plt
from arima import ArimaModel

# Cargar datos (ejemplo con datos sintéticos)
np.random.seed(42)
data = np.cumsum(np.random.normal(0, 1, 100))  # Serie temporal sintética

# Crear y entrenar modelo ARIMA
model = ArimaModel(data=data, order=(1, 1, 1))
model.fit()

# Evaluar el modelo
metrics = model.evaluate()
print("Métricas de evaluación:")
for key, value in metrics.items():
    print(f"{key}: {value}")

# Realizar predicciones
predictions = model.predict(steps=10)
print("\nPredicciones:")
print(predictions)

# Visualizar diagnósticos
model.plot_diagnostics()

# Visualizar pronóstico
model.plot_forecast(steps=10)

# Buscar mejores parámetros
best_order = model.grid_search(p_range=(0, 2), d_range=(0, 2), q_range=(0, 2))
print(f"\nMejor orden encontrado: {best_order}")

# Guardar resultados
model.to_csv("arima_results.csv")
\end{lstlisting}

\section{Ejemplo de Uso de SARIMA}
A continuación se muestra un ejemplo básico de cómo utilizar la clase \texttt{SarimaModel} para analizar y predecir una serie temporal con componentes estacionales:

\begin{lstlisting}[language=python]
import numpy as np
import pandas as pd
import matplotlib.pyplot as plt
from sarima import SarimaModel

# Cargar datos (ejemplo con datos sintéticos con estacionalidad)
np.random.seed(42)
t = np.arange(100)
trend = 0.1 * t
seasonal = 5 * np.sin(2 * np.pi * t / 12)  # Estacionalidad con período 12
noise = np.random.normal(0, 1, 100)
data = trend + seasonal + noise  # Serie temporal sintética con estacionalidad

# Crear y entrenar modelo SARIMA
model = SarimaModel(data=data, order=(1, 0, 1), seasonal_order=(1, 0, 1, 12))
model.fit()

# Evaluar el modelo
metrics = model.evaluate()
print("Métricas de evaluación:")
for key, value in metrics.items():
    print(f"{key}: {value}")

# Realizar predicciones
predictions = model.predict(steps=24)  # Predecir 2 períodos estacionales
print("\nPredicciones:")
print(predictions)

# Visualizar diagnósticos
model.plot_diagnostics()

# Visualizar pronóstico
model.plot_forecast(steps=24)

# Descomponer la serie
decomposition = model.decompose(type='additive')

# Buscar mejores parámetros (búsqueda limitada para el ejemplo)
best_params = model.grid_search(
    p_range=(0, 1), d_range=(0, 1), q_range=(0, 1),
    P_range=(0, 1), D_range=(0, 1), Q_range=(0, 1),
    s_values=[12]
)
print(f"\nMejores parámetros encontrados: {best_params}")

# Guardar resultados
model.to_csv("sarima_results.csv")
\end{lstlisting}

\chapter{Comparación entre ARIMA y SARIMA}

\section{Similitudes}
Los modelos ARIMA y SARIMA comparten varias similitudes:

\begin{itemize}
    \item Ambos son modelos estadísticos para el análisis y predicción de series temporales.
    \item Ambos utilizan componentes autorregresivos (AR) y de media móvil (MA).
    \item Ambos pueden aplicar diferenciación para hacer que la serie sea estacionaria.
    \item Ambos proporcionan métodos para ajuste, predicción, evaluación y diagnóstico.
    \item Ambos utilizan criterios como AIC y BIC para evaluar la calidad del modelo.
\end{itemize}

\section{Diferencias}
Las principales diferencias entre ARIMA y SARIMA son:

\begin{itemize}
    \item SARIMA incluye componentes estacionales (SAR, SI, SMA) que ARIMA no tiene.
    \item SARIMA es más adecuado para series temporales con patrones estacionales recurrentes.
    \item SARIMA tiene más parámetros para ajustar (p, d, q, P, D, Q, s).
    \item SARIMA incluye un método adicional (\texttt{decompose}) para descomponer la serie en tendencia, estacionalidad y residuos.
    \item La búsqueda en malla para SARIMA es más compleja y computacionalmente intensiva debido al mayor número de parámetros.
\end{itemize}

\section{Cuándo Usar Cada Modelo}
\begin{itemize}
    \item \textbf{Use ARIMA cuando:}
    \begin{itemize}
        \item La serie temporal no muestra patrones estacionales claros.
        \item Se necesita un modelo más simple con menos parámetros.
        \item El tiempo de cómputo es una preocupación.
    \end{itemize}

    \item \textbf{Use SARIMA cuando:}
    \begin{itemize}
        \item La serie temporal muestra patrones estacionales claros.
        \item Se necesita capturar y modelar la estacionalidad explícitamente.
        \item Se requiere descomponer la serie en sus componentes.
        \item La precisión en la predicción de patrones estacionales es importante.
    \end{itemize}
\end{itemize}

\chapter{Conclusiones}

\section{Resumen}
En este documento, hemos proporcionado una documentación detallada de las implementaciones de los modelos ARIMA y SARIMA para el análisis y predicción de series temporales. Estas implementaciones ofrecen una interfaz completa y fácil de usar para trabajar con estos modelos estadísticos.

Las clases \texttt{ArimaModel} y \texttt{SarimaModel} encapsulan toda la funcionalidad necesaria para ajustar modelos, realizar predicciones, evaluar el rendimiento, visualizar diagnósticos y resultados, optimizar hiperparámetros y exportar resultados.

\section{Aplicaciones Prácticas}
Los modelos ARIMA y SARIMA tienen numerosas aplicaciones prácticas en diversos campos:

\begin{itemize}
    \item \textbf{Finanzas:} Predicción de precios de acciones, tasas de interés y otros indicadores financieros.
    \item \textbf{Economía:} Pronóstico de indicadores económicos como PIB, inflación y desempleo.
    \item \textbf{Meteorología:} Predicción de temperaturas, precipitaciones y otros fenómenos climáticos.
    \item \textbf{Ventas:} Pronóstico de ventas con patrones estacionales.
    \item \textbf{Energía:} Predicción de consumo de energía y demanda eléctrica.
    \item \textbf{Transporte:} Análisis de patrones de tráfico y demanda de transporte.
    \item \textbf{Salud:} Predicción de incidencia de enfermedades estacionales.
\end{itemize}

\section{Limitaciones y Consideraciones}
A pesar de su utilidad, es importante tener en cuenta algunas limitaciones y consideraciones al utilizar estos modelos:

\begin{itemize}
    \item Los modelos ARIMA y SARIMA asumen que los patrones históricos continuarán en el futuro.
    \item La precisión de las predicciones tiende a disminuir a medida que el horizonte de predicción aumenta.
    \item La selección adecuada de los parámetros (p, d, q, P, D, Q, s) es crucial para el rendimiento del modelo.
    \item Estos modelos pueden no ser adecuados para series temporales con cambios estructurales o no linealidades complejas.
    \item La búsqueda en malla para optimizar hiperparámetros puede ser computacionalmente intensiva, especialmente para SARIMA.
\end{itemize}

\section{Trabajo Futuro}
Algunas posibles mejoras y extensiones para estas implementaciones incluyen:

\begin{itemize}
    \item Implementación de validación cruzada para series temporales.
    \item Integración con otros modelos de series temporales como Prophet o modelos de aprendizaje profundo.
    \item Optimización del rendimiento para conjuntos de datos grandes.
    \item Implementación de métodos para manejar valores faltantes y outliers.
    \item Desarrollo de interfaces gráficas para facilitar el uso por parte de usuarios no técnicos.
\end{itemize}

En resumen, las implementaciones de ARIMA y SARIMA presentadas en este documento proporcionan herramientas poderosas y flexibles para el análisis y predicción de series temporales, con aplicaciones en una amplia variedad de campos.

\end{document}
